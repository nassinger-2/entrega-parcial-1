\documentclass[
%% Opções: [*] comente para remover; [>] passada para pacotes
  article,%% Tipo de documento: article, book, report, etc. [>]
  a4paper,%% Tamanho de papel: a4paper, letterpaper, etc. [>]
  12pt,%% Tamanho de fonte: 10pt, 11pt, 12pt, etc. [>]
  fleqn,%% Alinhamento de equações à esquerda (comente para centralizado) [>]
  oneside,%% Impressão: oneside (anverso) ou twoside (anverso e verso) [>]
  % twocolumn,%% Texto em duas colunas (comente para uma coluna) [>]
  chapter = TITLE,%% Títulos de capítulos em maiúsculas [*]
  section = TITLE,%% Títulos de seções (secundárias) em maiúsculas [*]
]{abntex2}

\usepackage{caption}
\usepackage{hyperref}
\usepackage[
  BibURLs = false,
  ABNTNum = none
]{unoesc-article}

\addbibresource{unoesc-article.bib}

\titulo{GESTÃO INTELIGENTE DE TRÁFEGO EM OBRAS: uma abordagem baseada em monitoramento em tempo real}

\autor{
  Gabriel Nassinger%
  \thanks{
    \affil{Bacharel em Sistemas de Informação; Unoesc; Chapecó}%
    \sep\email{g.nassinger@unoesc.edu.br}%
  }
}

\data{\today}

\makeindex
\crefname{figure}{Figura}{Figuras}
\crefname{table}{Tabela}{Tabelas}

\begin{document}

\pretextual

\begin{paginadetitulo}

    \begin{ambienteresumo}
A gestão do tráfego em áreas submetidas a intervenções urbanas constitui um desafio crescente, especialmente em centros urbanos com grande fluxo de veículos. A execução de obras, muitas vezes realizada sem planejamento adequado, resulta em congestionamentos, atrasos e riscos à segurança de motoristas e pedestres. Este trabalho apresenta uma proposta de sistema inteligente baseado em monitoramento em tempo real, integrando sensores, câmeras e algoritmos de análise, para otimizar decisões relacionadas ao fechamento de vias e ao redirecionamento do tráfego. A utilização de tecnologias emergentes demonstra potencial para reduzir impactos negativos, aumentar a eficiência operacional e promover melhores condições de mobilidade urbana. 
\palavraschave{Gestão de tráfego. Monitoramento em tempo real. Mobilidade urbana. Sistemas inteligentes. Obras urbanas.}
    \end{ambienteresumo}

\end{paginadetitulo}

\textual

%% Introdução
\section{Introdução}

O crescimento urbano tem intensificado a demanda por intervenções estruturais em infraestruturas viárias. Contudo, o gerenciamento inadequado do tráfego durante obras continua sendo um dos principais fatores responsáveis por congestionamentos e incidentes urbanos. A ausência de recursos automatizados e de processos sistematizados torna as decisões operacionais lentas, subjetivas e pouco eficientes.

Com o avanço de tecnologias como sensores IoT, visão computacional e análise preditiva, surge a possibilidade de implementar sistemas mais inteligentes para o controle do tráfego. Essas ferramentas permitem que gestões municipais e equipes técnicas acessem dados em tempo real, capazes de melhorar a tomada de decisão durante obras, garantindo maior segurança, fluidez e previsibilidade.

\section{Delimitação do Tema}

Este estudo delimita-se à investigação e à aplicação de tecnologias de monitoramento em tempo real voltadas ao controle do tráfego urbano em áreas afetadas por obras. O foco está na integração entre dispositivos de coleta de dados, sistemas de análise e mecanismos de comunicação operacional para otimizar o fechamento de vias e a sinalização temporária.

\section{Justificativa}

A execução de obras urbanas interfere diretamente na mobilidade, afetando não apenas motoristas, mas também o transporte público, ciclistas e pedestres. Métodos tradicionais de gestão, como sinalização manual e intervenções pontuais, mostram-se insuficientes diante da complexidade atual do tráfego.

A adoção de sistemas inteligentes apresenta benefícios como:
\begin{itemize}
    \item redução de congestionamentos;
    \item melhoria da segurança viária durante intervenções;
    \item aumento da eficiência operacional das equipes de obra;
    \item fornecimento de dados para planejamento estratégico;
    \item maior transparência e comunicação com a população.
\end{itemize}

Assim, justifica-se a pesquisa pela necessidade de modernização e aprimoramento dos processos de gestão urbana.

\section{Objetivo Geral}

Desenvolver um sistema inteligente de monitoramento e análise em tempo real para auxiliar na gestão do tráfego em obras urbanas, permitindo tomadas de decisão mais eficientes no fechamento e na reabertura de vias, com foco em segurança e fluidez.

\subsection{Objetivos Específicos}

\begin{itemize}
    \item Mapear os principais problemas causados por obras na mobilidade urbana.
    \item Analisar tecnologias aplicáveis ao monitoramento de tráfego em tempo real.
    \item Desenvolver um protótipo funcional utilizando sensores e câmeras.
    \item Avaliar a efetividade do sistema por meio de simulações ou estudos de caso.
\end{itemize}

\postextual

\newpage
\printbibliography

\end{document}
